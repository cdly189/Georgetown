\documentclass[12pt]{article}
\usepackage{wasysym}
% for bold math symbols, \bm command:
\usepackage{bm}
% for real number set symbol, \mathcal{R}:
\usepackage{amsfonts}
\usepackage{color}
\newcommand{\tr}{^{\sf T}}
\newcommand{\m}[1]{{\bf{#1}}}
\newcommand{\g}[1]{\bm #1}
\newcommand{\C}[1]{{\cal {#1}}}
\topmargin -1.1in
%\textwidth 6.42in
\textwidth 7.00in
\textheight 10in
%\oddsidemargin -.075in
%\evensidemargin -.075in
\oddsidemargin -.30in
\evensidemargin -.30in
%-------------------------------------------------------------------------------
% get epsf.tex file, for encapsulated postscript files:
\input epsf
%-------------------------------------------------------------------------------
% macro for Postscript figures the easy way
% usage:  \postscript{file.ps}{scale}
% where scale is 1.0 for 100%, 0.5 for 50% reduction, etc.
%
\newcommand{\postscript}[2]
 {\setlength{\epsfxsize}{#2\hsize}
  \centerline{\epsfbox{#1}}}
\begin{document}
%----------------------------
\vspace{.2 cm}

\begin{center}
{ \bf Homework 2 :: MATH 504 ::  Due Friday, January 28th, 11:59 pm} \\[.2in]
\end{center}
Your homework submission must be a single pdf called ``LASTNAME-hw1.pdf" 
with your solutions to all theory problem to receive full credit. All answers must be typed in Latex.
Submission should be done on Canvas.


\begin{enumerate}

 \item Find the eigenvalues and the corresponding eigenvectors of the matrix
\[
A=
\left[
\begin{array}{ccc}
2 & 2 &1\\
1&3&1\\
1&2&2
\end{array}
\right].
\]
Determine the $\det(A)$ and $trace(A)$ using the eigenvalues.
%%%%

%\item Find the eigenvalue decomposition of the following matrix:
%\[
%A=
%\left[
%\begin{array}{ccc}
%3& 1 &0\\
%1&2&1\\
%0&1&3
%\end{array}
%\right]
%\]

\item  Consider the quadratic function
\[
f(x_1,x_2,x_3)=x_1^2+x_2^2+5x_3^2+2x_1x_2 -2x_1x_3+4x_2x_3+x_1-x_2.
\]
(a) Choose a matrix $A$ and vector $b$ so that with $x = (x_1, x_2,x_3)$, $f(x) = x^TAx + b \cdot x$.  
\\
(b) Choose another matrix $B$, such that $A \neq B$ and $B=B^T$ so that $f(x) = x^TBx + b^Tx$. \\
%(The point here is that the matrix $A$ is not unique.) \\
%(c) Determine whether the function $f$ is convex, concave, or saddle. \\
%(d) Find the extreme point of $f$ using the Gauss-Jordan method if exists. \\
%(d) Determine the Hessian of $f$.
(c) Determine the gradient vector for $f$.
\end{enumerate}



\end{document}